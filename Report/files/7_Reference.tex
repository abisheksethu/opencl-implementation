\begin{thebibliography}{1}

	\bibitem{1}H. Esmaeilzadeh, E. Blem, R. St. Amant, K. Sankaralingam, and D. Burger, \emph{Dark silicon and the end of multicore scaling}, Micro, IEEE, 32(3):122-134, May 2012.
	
	\bibitem{2}A. R. Brodtkorb, C. Dyken, T. R. Hagen, J. M. Hjelmervik, and O. O. Storaasli, \emph{State-of-the-art in heterogeneous computing.} Scientific Programming, 18(1):1-33, 2010.

	\bibitem{3}B. D. de Dinechin, D. V. Amstel, M. Poulhies, and G. Lager,\emph{Time-critical computing on a single-chip massively parallel processor.} In Proceedings of the Design, Automation and Test in Europe Conference(DATE), pages 97:197:6, 2014 
	
	\bibitem{4}B. D. de Dinechin, R. Ayrignac, P. Beaucamps, P. Couvert, B. Ganne, P. G. de Massas, F. Jacquet, S. Jones, N. M. Chaisemartin, F. Riss, \emph{A clustered manycore processor architecture for embedded and accelerated applications.} In Proceedings of the International Conference on High Performance Extreme Computing Conference (HPEC), 2013
	
	\bibitem{5}L. Gwennap, \emph{Adapteva: More ops, less watts.} Microprocessor Report, 6(13):11-02, 2011

	\bibitem{6}A. Varghese, B. Edwards, G. Mitra, and A. P. Rendell, \emph{Programming the Adapteva Epiphany 64-core network-on-chip coprocessor.} In Parallel Distributed Processing Symposium Workshops (IPDPSW), pages 984-992, May 2014
	
	\bibitem{7}A. Hodjat and I. Verbauwhede,\emph{A 21.54 gbits/s fully pipelined AES processor on FPGA.} In IEEE Symposium on FPGAs for Custom Computing Machines (FCCM), pages 308-309. IEEE, 2004.
	
	\bibitem{8}A. Descampe, F. Devaux, G. Rouvroy, B. Macq, and J. Legat,\emph{An efficient FPGA implementation of a exible JPEG2000 decoder for digital cinema}, In European Signal Processing Conference, pages 2019-2022. IEEE, 2004.
	
	\bibitem{9}O. T. Albaharna, P. Y. K. Cheung, and T. J. Clarke, \emph{On the viability of FPGA-based integrated coprocessors}, In IEEE Symposium on Field-Programmable Custom Computing Machines (FCCM), pages 206-215, 1996.
	
	\bibitem{10}S. Ahmad, V. Boppana, I. Ganusov, V. Kathail, V. Rajagopalan, and R. Wittig, \emph{A 16-nm multiprocessing system-on-chip Field-programmable gate array platform},IEEE Micro, 36(2):48-62, 2016.
	
	\bibitem{11}Xilinx Ltd. Zynq-7000 technical reference manual, \url{http://www.xilinx.com/support/documentation/user_guides/ug585-Zynq-7000-TRM.pdf}, 2013
	
	\bibitem{12}The OpenCL Specification, Version 1.2, Khronos OpenCL Working Group, Specification, Rev. 19, November 2012, \url{https://www.khronos.org/registry/cl/specs/opencl-1.2.pdf}
		
	\bibitem{13}Parker, Samuel J, \emph{An automated OpenCL FPGA compilation framework targeting a configurable, VLIW chip multiprocessor}
	
	\bibitem{14}Hugo van der Wijst, \emph{An Accelerator based on the $\uprho$-VEX Processor: An Exploration using OpenCL}
	
	\bibitem{15}Kernel De-SPMD Compilation for Texas Instruments' DSPs, \url{http://portablecl.org/texas-instruments-pocl-use-case.html}
	
	\bibitem{16}Lattner, C., Adve, V. \emph{LLVM: A compilation framework for lifelong program analysis and transformation.} In: Proceedings of International Symposium on Code Generation Optimization, p. 75 (2004)
	
	\bibitem{17}Khronos Group: SPIR 1.2 Specification for OpenCL (2014)
	
	\bibitem{18}Pekka Jaaskelainen, Carlos Sánchez de La Lama, Erik Schnetter, Kalle Raiskila, Jarmo Takala, Heikki Berg, \emph{pocl: A Performance-Portable OpenCL Implementation},International Journal of Parallel Programming, Springer. October 2015, Volume 43, Issue 5.
	
	\bibitem{19}\emph{LLVM compiler infrastructure}, \url{http://llvm.org/}
	
	\bibitem{20}\emph{Clang: A C language frontend for LLVM.}, \url{http://clang.llvm.org/}
	
	\bibitem{21}\emph{Xillybus IP core}, \url{http://xillybus.com/downloads/xillybus_product_brief.pdf}
	
	\bibitem{22}\emph{Xillybus FPGA Interface}, \url{http://xillybus.com/downloads/doc/xillybus_getting_started_zynq.pdf}
	
	\bibitem{23}\emph{Xillybus Host Interface}, \url{http://xillybus.com/downloads/doc/xillybus_getting_started_linux.pdf}	
	
\end{thebibliography}