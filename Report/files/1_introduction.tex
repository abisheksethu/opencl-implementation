\chapter{Introduction}
\label{ch1_introduction}
The open computing language (OpenCL) has become an industry standard for parallel programming on a range of heterogeneous platforms including CPUs, GPUs, FPGAs, and other accelerators. One of the main reason is that the OpenCL has the benefit of being portable across architectures without changes to algorithm source code.
FPGA vendors have recently started focusing on OpenCL for FPGAs because of its ability to leverage the parallelism inherent to heterogeneous computing platforms. OpenCL allows programs running on a host computer to launch accelerator kernels which can be compiled at run-time for a specific architecture, thus enabling portability.
In OpenCL, parallelism is explicitly specified by the programmer, and compilers can use system information at runtime to scale the performance of the application by executing multiple replicated copies of the application kernel in hardware.

FPGA vendors have recently released OpenCL based tools (Altera OpenCL and Xilinx SDAccel) to bridge the gap between the expressiveness of sequential programming languages and the parallel capabilities of the FPGA hardware. The reason for this is the introduction of the more capable heterogeneous system on chip (SoC) platforms/hybrid FPGAs which tightly couple general-purpose processors with high-performance FPGA fabrics and provide a more energy efficient alternative to high-performance CPUs and/or GPUs
within the tight power budget required by high performance embedded systems. Hence techniques for mapping OpenCL kernels to FPGA hardware have attracted both academic and industrial attention in the last few years.

However, it is a challenge to execute OpenCL applications on the platforms that are not supported by the vendors. Currently, one such example is Xilinx Zynq platform which combines dual-core ARM Cortex-A9 processors with high-performance FPGA fabric. The FPGA fabric is normally used to host accelerators and those should be able to execute OpenCL kernels. Exposing the accelerators (residing on the FPGA fabric) as an OpenCL device is necessary for executing OpenCL applications on Zynq platform.

\section{Motivation}
The open source community has developed many software frameworks for OpenCL implementation. Portable Computing Language (POCL) aims to become an MIT licensed OpenCL standard. POCL can be easily portable for CPU, heterogeneous GPUs and accelerators. In this thesis work, we explore POCL software framework to expose the FPGA accelerators as an OpenCL device.

\section{Contribution}
The contributions are summarized as follows:
\begin{itemize} 
	\item A technique for exposing the FPGA accelerator as an OpenCL device using pocl software framework.
	\item Integration of OpenCL drivers in device layer of POCL using xillybus Linux driver.
	\item Experiments to test the OpenCL APIs for data transfer between memory, processor and the accelerator.
\end{itemize}

\section{Organization}
The thesis is organized as follows. Chapter 2 discusses background and previous work on implementation of OpenCL standard. In Chapter 3, we describe the concepts of OpenCL Implementation using POCL and its relation to POCL software framework. Chapter 4 discusses the advantage of xillybus data transfer APIs that can be used within OpenCL driver. In chapter 5, we execute an OpenCL application on Zynq and profile the OpenCL APIs for data transfer between memory, processor and the accelerator. Finally, chapter 6 concludes the thesis with future work.